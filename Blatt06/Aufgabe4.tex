\documentclass{article}

\usepackage[utf8]{inputenc} \usepackage[ngerman]{babel}
\usepackage{graphicx}
\usepackage{amssymb} \usepackage{amsmath}
\usepackage{listings}
\usepackage{color}
\usepackage{latexsym}

\title{Aufgabenblatt 6 - Aufgabe 4}

\author{}

\begin{document}
\maketitle

\begin{enumerate}
\item[(a)]
Behauptung:
Es wird das Gewicht einer Kante $e \in E \backslash E'$ erhöht,
$T$ ist auch ein minimaler Spannbaum im so veränderten Graphen $G'$.\\
Beweis:\\
Wenn $T$ kein minimaler Spannbaum in $G'$ sei, muss es einen minimalen Spannbaum $T' \neq T$ geben.\\
Wenn $e \in T'$, dann $w(T')$ in $G'$ gr{\"o}{\ss}er als in $G$ und muss somit gr{\"o}{\ss}er als $T$ sein.\\
Wenn $e \notin T'$, dann sind sowohl $T$ als auch $T'$ von der Ver{\"a}nderung unbeeinflusst, somit bleibt $T$ auf jeden Fall ein minimaler Spannbaum.
\item[(b)]
Wenn $e \in E'$ ist und $e$ um $\Delta w$ verringert wird, wird auch $w(T)$ um $\Delta w$ verringert. Da die Gewichtsumme jedes weiteren möglichen Spannbaumes ebenfalls maximal um $\Delta w$ verringert werden kann, bleibt $T$ auf jeden Fall ein minimaler Spannbaum. \\ 
Wenn $e \notin E'$ führen wir Kruskals Algorithmus wie bei der Berechnung von $T$ aus, nur dass wir zu Beginn alle Kanten $d$ mit $w'(d)<w'(e)$, die somit unverändert blieben, überspringen können und erst ab da mit dem neuen Gewicht weiter rechnen.
\end{enumerate}
\end{document}