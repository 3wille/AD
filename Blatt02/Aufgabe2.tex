\documentclass{article}

\usepackage[utf8]{inputenc} \usepackage[ngerman]{babel}

\usepackage{amssymb} \usepackage{amsmath}

\usepackage{latexsym}

\title{Aufgabenblatt 2 - Aufgabe 2}

\author{}

\begin{document}

\maketitle

\begin{enumerate}
\item[(a)]
$T(n) = 2T\left(\frac{n}{2}\right)+n^0$\\
Die Lauftzeit ist $O(n)$, da pro rekursivem Aufruf nur Operationen mit $\theta\left(1\right)$ durchgeführt werden.
\item[(b)]
Im best-case kann die Laufzeit auf $\theta\left(1\right)$ verbessert werden.
\item[(c)]
ORDER1:
\begin{tabular}{|c|c|c|c|c|c|c|c|c|c|c|c|c|c|c|c|}
	\hline S & R & I & T & A & H & G & I & E & M & O & L & W & K & A & L \\
	\hline
\end{tabular}\\
\\
ORDER2:
\begin{tabular}{|c|c|c|c|c|c|c|c|c|c|c|c|c|c|c|c|}
	\hline A & T & I & H & R & I & G & E & L & O & W & M & A & K & L \\
	\hline
\end{tabular}\\
\\
ORDER3:
\begin{tabular}{|c|c|c|c|c|c|c|c|c|c|c|c|c|c|c|c|}
	\hline A & T & H & I & I & E & G & R & L & W & O & A & L & K & M \\
	\hline
\end{tabular}\\
\item[(d)]
ORDER2 für Univeristy:
\begin{tabular}{|c|c|c|c|c|c|c|c|c|c|c|c|c|c|c|c|}
	\hline R & V & S & N & E & I & T & U & I & Y \\
	\hline
\end{tabular}
\item[(e)]
ORDER3 für ternären Baum: \\
\begin{tabular}{|c|c|c|c|c|c|c|c|c|c|c|c|c|c|c|c|}
	\hline W & L & E & I & I & H & T & M & K & O & E & L & A & G & R & S \\
	\hline
\end{tabular}\\
\end{enumerate}
\end{document}

